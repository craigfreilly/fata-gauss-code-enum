% vim: set spell spelllang=en tw=100 et sw=4 sts=4 foldmethod=marker foldmarker={{{,}}} :

\documentclass{beamer}

\usepackage{tikz}
\usepackage{xcolor}
\usepackage{complexity}
\usepackage{hyperref}
\usepackage{microtype}
\usepackage{amsmath}                   % \operatorname
\usepackage{amsfonts}                  % \mathcal
\usepackage{amssymb}                   % \nexists
\usepackage{gnuplot-lua-tikz}          % graphs
\usepackage[vlined]{algorithm2e} % algorithms
\usepackage{centernot}
\usepackage{mathtools}
\usepackage{listings}

\usetikzlibrary{shapes, arrows, shadows, calc, positioning, fit}
\usetikzlibrary{decorations.pathreplacing, decorations.pathmorphing, shapes.misc}
\usetikzlibrary{tikzmark}

\definecolor{uofguniversityblue}{rgb}{0, 0.219608, 0.396078}

\definecolor{uofgheather}{rgb}{0.356863, 0.32549, 0.490196}
\definecolor{uofgaquamarine}{rgb}{0.603922, 0.72549, 0.678431}
\definecolor{uofgslate}{rgb}{0.309804, 0.34902, 0.380392}
\definecolor{uofgrose}{rgb}{0.823529, 0.470588, 0.709804}
\definecolor{uofgmocha}{rgb}{0.709804, 0.564706, 0.47451}
\definecolor{uofgsandstone}{rgb}{0.321569, 0.278431, 0.231373}
\definecolor{uofgforest}{rgb}{0, 0.2, 0.129412}
\definecolor{uofglawn}{rgb}{0.517647, 0.741176, 0}
\definecolor{uofgcobalt}{rgb}{0, 0.615686, 0.92549}
\definecolor{uofgturquoise}{rgb}{0, 0.709804, 0.819608}
\definecolor{uofgsunshine}{rgb}{1.0, 0.862745, 0.211765}
\definecolor{uofgpumpkin}{rgb}{1.0, 0.72549, 0.282353}
\definecolor{uofgthistle}{rgb}{0.584314, 0.070588, 0.447059}
\definecolor{uofgrust}{rgb}{0.603922, 0.227451, 0.023529}
\definecolor{uofgburgundy}{rgb}{0.490196, 0.133333, 0.223529}
\definecolor{uofgpillarbox}{rgb}{0.701961, 0.047059, 0}
\definecolor{uofglavendar}{rgb}{0.356863, 0.301961, 0.580392}

\tikzset{vertex/.style={draw, circle, inner sep=0pt, minimum size=0.5cm, font=\small\bfseries}}
\tikzset{notvertex/.style={vertex, color=white, text=black}}
\tikzset{plainvertex/.style={vertex}}
\tikzset{vertexc1/.style={vertex, fill=uofgburgundy, text=white}}
\tikzset{vertexc2/.style={vertex, fill=uofgsandstone, text=white}}
\tikzset{vertexc3/.style={vertex, fill=uofgforest, text=white}}
\tikzset{vertexc4/.style={vertex, fill=uofgheather, text=white}}
\tikzset{edge/.style={color=black!50!white}}
\tikzset{bedge/.style={ultra thick}}
\tikzset{edged/.style={color=screengrey, dashed}}
\tikzset{edgel3/.style={color=uofgrose, ultra thick}}

% {{{ theme things
\useoutertheme[footline=authortitle]{miniframes}
\useinnertheme{rectangles}

\setbeamerfont{block title}{size={}}
\setbeamerfont{title}{size=\large,series=\bfseries}
\setbeamerfont{section title}{size=\large,series=\mdseries}
\setbeamerfont{author}{size=\normalsize,series=\mdseries}
\setbeamercolor*{structure}{fg=uofguniversityblue}
\setbeamercolor*{palette primary}{use=structure,fg=black,bg=white}
\setbeamercolor*{palette secondary}{use=structure,fg=white,bg=uofgcobalt}
\setbeamercolor*{palette tertiary}{use=structure,fg=white,bg=uofguniversityblue}
\setbeamercolor*{palette quaternary}{fg=white,bg=black}

\setbeamercolor*{titlelike}{parent=palette primary}

\beamertemplatenavigationsymbolsempty

\setbeamertemplate{title page}
{
    \begin{tikzpicture}[remember picture, overlay]
        \node at (current page.north west) {
            \begin{tikzpicture}[remember picture, overlay]
                \fill [fill=uofguniversityblue, anchor=north west] (0, 0) rectangle (\paperwidth, -2.6cm);
            \end{tikzpicture}
        };

        \node (logo) [anchor=north east, shift={(-0.6cm,-0.6cm)}] at (current page.north east) {
            \includegraphics*[keepaspectratio=true,scale=0.7]{UoG_keyline.pdf}
        };

        \node [anchor=west, xshift=0.2cm] at (current page.west |- logo.west) {
            \begin{minipage}{0.65\paperwidth}\raggedright
                {\usebeamerfont{title}\usebeamercolor[white]{}\inserttitle}\\[0.1cm]
                {\usebeamerfont{author}\usebeamercolor[white]{}\insertauthor}
            \end{minipage}
        };
    \end{tikzpicture}
}

\setbeamertemplate{section page}
{
    \begin{centering}
        \begin{beamercolorbox}[sep=12pt,center]{part title}
            \usebeamerfont{section title}\insertsection\par
        \end{beamercolorbox}
    \end{centering}
}

\newcommand{\frameofframes}{/}
\newcommand{\setframeofframes}[1]{\renewcommand{\frameofframes}{#1}}

\makeatletter
\setbeamertemplate{footline}
{%
    \begin{beamercolorbox}[colsep=1.5pt]{upper separation line foot}
    \end{beamercolorbox}
    \begin{beamercolorbox}[ht=2.5ex,dp=1.125ex,%
        leftskip=.3cm,rightskip=.3cm plus1fil]{author in head/foot}%
        \leavevmode{\usebeamerfont{author in head/foot}\insertshortauthor}%
        \hfill%
        {\usebeamerfont{institute in head/foot}\usebeamercolor[fg]{institute in head/foot}\insertshortinstitute}%
    \end{beamercolorbox}%
    \begin{beamercolorbox}[ht=2.5ex,dp=1.125ex,%
        leftskip=.3cm,rightskip=.3cm plus1fil]{title in head/foot}%
        {\usebeamerfont{title in head/foot}\insertshorttitle}%
        \hfill%
        {\usebeamerfont{frame number}\usebeamercolor[fg]{frame number}\insertframenumber~\frameofframes~\inserttotalframenumber}
    \end{beamercolorbox}%
    \begin{beamercolorbox}[colsep=1.5pt]{lower separation line foot}
    \end{beamercolorbox}
}

% }}}

\title{Enumeration of (unique reduced alternating) knot diagrams}
\author[Craig Reilly]{\textbf{Craig Reilly}}

\begin{document}

{
    \usebackgroundtemplate{
        \tikz[overlay, remember picture]
        \node[at=(current page.south), anchor=south, inner sep=0pt]{\includegraphics*[keepaspectratio=true, width=\paperwidth]{background.jpg}};
    }
    \begin{frame}[plain,noframenumbering]
        \titlepage
    \end{frame}
}

\begin{frame}{Knots}
    \begin{itemize}
        \item A \emph{global constraint} is one which can operate on arbitrarily many variables.
    \end{itemize}
\end{frame}

% \begin{frame}{Two-Colouring a Triangle}
% 
%     \begin{columns}
%         \begin{column}{0.45\textwidth}
%             \begin{align*}
%                 & x_1 \in \{ 0, 1 \} \\
%                 & x_2 \in \{ 0, 1 \} \\
%                 & x_3 \in \{ 0, 1 \} \\
%                 & \mathrlap{\textit{alldifferent}( x_1, x_2, x_3 )} \\
%                 \\
%                 \\
%             \end{align*}
%         \end{column}
%         \begin{column}{0.45\textwidth}
%             \begin{tikzpicture}
%                 \node [draw, ellipse] (C1) at (90:1.5) { $\{0,1\}$ };
%                 \node [draw, ellipse] (C2) at (210:1.5) { $\{0,1\}$ };
%                 \node [draw, ellipse] (C3) at (330:1.5) {  $\{0,1\}$  };
% 
%                 \node [anchor = south] (C) at (0, 0) { $\textit{alldifferent}$ };
% 
%                 \draw (C) to (C1);
%                 \draw (C) to (C2);
%                 \draw (C) to (C3);
%             \end{tikzpicture}
%         \end{column}
%     \end{columns}
% 
% \end{frame}
% 
% \begin{frame}{Decomposing ``All Different''}
% 
%     \begin{columns}
%         \begin{column}{0.45\textwidth}
%             \begin{align*}
%                 & x_1 \in \{ 0, 1 \} \\
%                 & x_2 \in \{ 0, 1 \} \\
%                 & x_3 \in \{ 0, 1 \} \\
%                 & x_1 \ne x_2 \\
%                 & x_1 \ne x_3 \\
%                 & x_2 \ne x_3 \\
%             \end{align*}
%         \end{column}
%         \begin{column}{0.45\textwidth}
%             \begin{tikzpicture}
%                 \node [draw, ellipse] (C1) at (90:1.5) { $\{0,1\}$ };
%                 \node [draw, ellipse] (C2) at (210:1.5) { $\{0,1\}$ };
%                 \node [draw, ellipse] (C3) at (330:1.5) {  $\{0,1\}$  };
% 
%                 \draw (C1) to node [xshift=-0.7em] { $\ne$ } (C2);
%                 \draw (C1) to node [xshift=0.7em] { $\ne$ } (C3);
%                 \draw (C2) to node [yshift=0.7em] { $\ne$ } (C3);
%             \end{tikzpicture}
%         \end{column}
%     \end{columns}
% 
% \end{frame}
% 
% \begin{frame}{What Does Propagation Do?}
% 
%     \begin{itemize}
%         \item Let's consider the constraint $x_1 \ne x_2$.
%             \begin{itemize}
%                 \item If $x_1 = 0$, we can give $x_2 = 1$, so that's OK.
%                 \item If $x_1 = 1$, we can give $x_2 = 0$, so that's OK.
%                 \item If $x_2 = 0$, we can give $x_1 = 1$, so that's OK.
%                 \item If $x_2 = 1$, we can give $x_1 = 0$, so that's OK.
%             \end{itemize}
%         \item Let's consider the constraint $x_1 \ne x_3$.
%             \begin{itemize}
%                 \item etc
%             \end{itemize}
%         \item Let's consider the constraint $x_2 \ne x_3$.
%             \begin{itemize}
%                 \item etc
%             \end{itemize}
%         \item So no values are deleted, and everything looks OK.
%         \item Actually, there's a more efficient algorithm: $\ne$ won't do anything unless one of
%             the variables only has one value. Some solvers won't trigger the constraint unless this
%             happens.
%     \end{itemize}
% 
% \end{frame}
% 
% \begin{frame}{What Would a Human Do?}
%     \begin{center}
%         ``Duh, obviously there's no solution! There \\ aren't enough numbers to go around.''
%     \end{center}
% 
%     \begin{itemize}
%         \item Unfortunately ``stare at it for a few seconds then write down the answer'' is not an algorithm.
% 
%         \item But if we don't decompose the constraint, we \emph{can} come up with a propagator
%             which can tell that there's no solution.
%     \end{itemize}
% \end{frame}
% 
% \begin{frame}{Matchings}
% 
%     \begin{columns}
%         \begin{column}{0.70\textwidth}
%             \begin{itemize}
%                 \item Draw a vertex on the left for each variable, and a vertex on the right for each value.
%                 \item Draw edges from each variable to each of its values.
%                 \item A \emph{maximum cardinality matching} is where you pick as many edges as
%                     possible, but each vertex can only be used at most once.
%                 \item We can find this in polynomial time (see Algorithmics II).
%                 \item There is a matching which covers each variable if and only if the constraint
%                     can be satisfied.
%             \end{itemize}
%         \end{column}
%         \begin{column}{0.28\textwidth}
%             \begin{tikzpicture}
%                 \node (X1) at (0, 2) { $x_1$ };
%                 \node (X2) at (0, 1) { $x_2$ };
%                 \node (X3) at (0, 0) { $x_3$ };
% 
%                 \node (V0) at (2, 2) { $0$ };
%                 \node (V1) at (2, 1) { $1$ };
%                 \node <3-> (V2) at (2, 0) { $2$ };
% 
%                 \draw <1, 3> [color=uofgsandstone!50] (X1) -- (V0);
%                 \draw <1, 3> [color=uofgsandstone!50] (X1) -- (V1);
%                 \draw <1, 3> [color=uofgsandstone!50] (X2) -- (V0);
%                 \draw <1, 3> [color=uofgsandstone!50] (X2) -- (V1);
%                 \draw <1, 3> [color=uofgsandstone!50] (X3) -- (V0);
%                 \draw <1, 3> [color=uofgsandstone!50] (X3) -- (V1);
%                 \draw <3> [color=uofgsandstone!50] (X2) -- (V2);
% 
%                 \draw <2> [color=uofgsandstone!50] (X1) -- (V1);
%                 \draw <2> [color=uofgsandstone!50] (X2) -- (V0);
%                 \draw <2> [color=uofgsandstone!50] (X3) -- (V0);
%                 \draw <2> [color=uofgsandstone!50] (X3) -- (V1);
%                 \draw <2> [ultra thick] (X1) -- (V0);
%                 \draw <2> [ultra thick] (X2) -- (V1);
% 
%                 \draw <4> [color=uofgsandstone!50] (X1) -- (V1);
%                 \draw <4> [color=uofgsandstone!50] (X2) -- (V0);
%                 \draw <4> [color=uofgsandstone!50] (X2) -- (V1);
%                 \draw <4> [color=uofgsandstone!50] (X3) -- (V0);
%                 \draw <4> [ultra thick] (X1) -- (V0);
%                 \draw <4> [ultra thick] (X3) -- (V1);
%                 \draw <4> [ultra thick] (X2) -- (V2);
%             \end{tikzpicture}
%         \end{column}
%     \end{columns}
% 
% \end{frame}
% 
% \begin{frame}{Sudoku}
% 
%     \centering
%     \includegraphics*[keepaspectratio=true,scale=0.18]{sudoku.png}
% 
% \end{frame}
% 
% \begin{frame}{How do Humans Solve Sudoku?}
%     \begin{center}\begin{tikzpicture}[scale=0.4]
%         \draw[thick, scale=3, color=uofgslate] (0, 0) grid (9, 1);
% 
%         \node <1>   [anchor=center] at (1.5, 1.5)  { 18 };
%         \node <2>   [anchor=center] at (1.5, 1.5)  { \textcolor{uofgcobalt}{1}8 };
%         \node <3>   [anchor=center] at (1.5, 1.5)  { \textcolor{uofgcobalt}{1} };
%         \node <4->  [anchor=center] at (1.5, 1.5)  { 1 };
% 
%         \node <1-4> [anchor=center] at (4.5, 1.5)  { 23 };
%         \node <5-6> [anchor=center] at (4.5, 1.5)  { \textcolor{uofgcobalt}{23} };
%         \node <7->  [anchor=center] at (4.5, 1.5)  { 23 };
% 
%         \node <1-4> [anchor=center] at (7.5, 1.5)  { 23 };
%         \node <5-6> [anchor=center] at (7.5, 1.5)  { \textcolor{uofgcobalt}{23} };
%         \node <7->  [anchor=center] at (7.5, 1.5)  { 23 };
% 
%         \node <1-5> [anchor=center] at (10.5, 1.5) { 245 };
%         \node <6>   [anchor=center] at (10.5, 1.5) { \textcolor{uofgpillarbox}{2}45 };
%         \node <7>   [anchor=center] at (10.5, 1.5) { 45 };
%         \node <8-9> [anchor=center] at (10.5, 1.5) { \textcolor{uofgcobalt}{45} };
%         \node <10-> [anchor=center] at (10.5, 1.5) { 45 };
% 
%         \node <1-7> [anchor=center] at (13.5, 1.5) { 456 };
%         \node <8-9>  [anchor=center] at (13.5, 1.5) { \textcolor{uofgcobalt}{456} };
%         \node <10-> [anchor=center] at (13.5, 1.5) { 456 };
% 
%         \node <1-7> [anchor=center] at (16.5, 1.5) { 456 };
%         \node <8-9> [anchor=center] at (16.5, 1.5) { \textcolor{uofgcobalt}{456} };
%         \node <10-> [anchor=center] at (16.5, 1.5) { 456 };
% 
%         \node <1-5> [anchor=center] at (19.5, 1.5) { 279 };
%         \node <6>   [anchor=center] at (19.5, 1.5) { \textcolor{uofgpillarbox}{2}79 };
%         \node <7->  [anchor=center] at (19.5, 1.5) { 79 };
% 
%         \node <1-5> [anchor=center] at (22.5, 1.5) { 378 };
%         \node <6>   [anchor=center] at (22.5, 1.5) { \textcolor{uofgpillarbox}{3}78 };
%         \node <7->  [anchor=center] at (22.5, 1.5) { 78 };
% 
%         \node <1-5> [anchor=center] at (25.5, 1.5) { 23589 };
%         \node <6>   [anchor=center] at (25.5, 1.5) { \textcolor{uofgpillarbox}{23}589 };
%         \node <7-8> [anchor=center] at (25.5, 1.5) { 589 };
%         \node <9>   [anchor=center] at (25.5, 1.5) { \textcolor{uofgpillarbox}{5}89 };
%         \node <10-> [anchor=center] at (25.5, 1.5) { 89 };
% 
%     \end{tikzpicture}\end{center}
% \end{frame}
% 
% \begin{frame}[fragile]{What Does Choco Do?}
%     \only<1> {
%         \lstinputlisting[language=Java, basicstyle=\scriptsize\ttfamily, keywordstyle=\color{uofgcobalt}]{code/OneRowSudoku-snippet.java}
%     }
% 
%     \only<2> {
%         \lstinputlisting[basicstyle=\scriptsize\ttfamily]{code/OneRowSudoku-output.txt}
% 
%         \begin{center}\begin{tikzpicture}[scale=0.4]
%             \draw[thick, scale=3, color=uofgslate] (0, 0) grid (9, 1);
% 
%             \node [anchor=center] at (1.5, 1.5)  { 1 };
%             \node [anchor=center] at (4.5, 1.5)  { 23 };
%             \node [anchor=center] at (7.5, 1.5)  { 23 };
%             \node [anchor=center] at (10.5, 1.5) { 45 };
%             \node [anchor=center] at (13.5, 1.5) { 456 };
%             \node [anchor=center] at (16.5, 1.5) { 456 };
%             \node [anchor=center] at (19.5, 1.5) { 79 };
%             \node [anchor=center] at (22.5, 1.5) { 78 };
%             \node [anchor=center] at (25.5, 1.5) { 89 };
% 
%         \end{tikzpicture}\end{center}
%     }
% 
%     \only<3> {
%         \lstinputlisting[language=Java, basicstyle=\scriptsize\ttfamily, keywordstyle=\color{uofgcobalt}]{code/OneRowSudoku-neq-snippet.java}
%     }
% 
%     \only<4> {
%         \lstinputlisting[basicstyle=\scriptsize\ttfamily]{code/OneRowSudoku-neq-output.txt}
%     }
% \end{frame}
% 
% \begin{frame}{Hall Sets}
%     \begin{itemize}
%         \item A \emph{Hall set} of size $n$ is a set of $n$ variables from an ``all different''
%             constraint, whose domains have $n$ values between them.
% 
%         \item If we can find a Hall set, we can safely remove these values from the domains of every
%             other variable involved in the constraint.
% 
%         \item If we do this for every Hall set, we delete every value that cannot appear in at least
%             one way of satisfying the constraint.
%     \end{itemize}
% \end{frame}
% 
% \begin{frame}{Only Occurs in One Place?}
% 
%     \begin{center}
%         ``But wait! We said that the value 1 only occurs \\
%         in one place. That doesn't sound like a Hall set!''
%     \end{center}
% 
%     \begin{itemize}
%         \item <2-> The ``only occurs in one place'' rule we used first is just a Hall set of size 8.
%     \end{itemize}
% \end{frame}
% 
% \begin{frame}{Is Choco Magic?}
% 
%     \only<1> {
%         \begin{itemize}
%             \item There are $2^n$ potential Hall sets, so considering them all is probably a bad
%                 idea. However, there is a polynomial algorithm.
% 
%             \item This algorithm isn't examinable, but here's an animation of roughly how it works,
%                 so you don't have to believe it's magic any more.
%         \end{itemize}
%     }
% 
%     \only <2-12> {
%         \begin{tikzpicture}[remember picture, overlay]
%             \node (logo) [anchor=north east, shift={(-0.4cm,-0.8cm)}] at (current page.north east) {
%                 \begin{tikzpicture}[scale=0.25, font=\tiny]
%                     \draw[thick, scale=3, color=uofgslate] (0, 0) grid (9, 1);
%                     \node <2-11> [anchor=center] at (1.5, 1.5)  { 18 };
%                     \node <2-11> [anchor=center] at (4.5, 1.5)  { 23 };
%                     \node <2-11> [anchor=center] at (7.5, 1.5)  { 23 };
%                     \node <2-11> [anchor=center] at (10.5, 1.5) { 245 };
%                     \node <2-11> [anchor=center] at (13.5, 1.5) { 456 };
%                     \node <2-11> [anchor=center] at (16.5, 1.5) { 456 };
%                     \node <2-11> [anchor=center] at (19.5, 1.5) { 279 };
%                     \node <2-11> [anchor=center] at (22.5, 1.5) { 378 };
%                     \node <2-11> [anchor=center] at (25.5, 1.5) { 23589 };
% 
%                     \node <12> [anchor=center] at (1.5, 1.5)  { \textcolor{uofgcobalt}{1}\textcolor{uofgpillarbox}{8} };
%                     \node <12> [anchor=center] at (4.5, 1.5)  { 23 };
%                     \node <12> [anchor=center] at (7.5, 1.5)  { 23 };
%                     \node <12> [anchor=center] at (10.5, 1.5) { \textcolor{uofgpillarbox}{2}45 };
%                     \node <12> [anchor=center] at (13.5, 1.5) { 456 };
%                     \node <12> [anchor=center] at (16.5, 1.5) { 456 };
%                     \node <12> [anchor=center] at (19.5, 1.5) { \textcolor{uofgpillarbox}{2}79 };
%                     \node <12> [anchor=center] at (22.5, 1.5) { \textcolor{uofgpillarbox}{3}78 };
%                     \node <12> [anchor=center] at (25.5, 1.5) { \textcolor{uofgpillarbox}{235}89 };
%                 \end{tikzpicture}
%             };
%         \end{tikzpicture}%
%         \centering
%         \begin{tikzpicture}[scale=0.75]
%             \path (-1, -0.6) rectangle (7, 8.6);
%             \node <3-> [inner sep = 1pt] (X1) at (0, 8) { $\mathit{row}[0]$ };
%             \node <3-> [inner sep = 1pt] (X2) at (0, 7) { $\mathit{row}[1]$ };
%             \node <3-> [inner sep = 1pt] (X3) at (0, 6) { $\mathit{row}[2]$ };
%             \node <3-> [inner sep = 1pt] (X4) at (0, 5) { $\mathit{row}[3]$ };
%             \node <3-> [inner sep = 1pt] (X5) at (0, 4) { $\mathit{row}[4]$ };
%             \node <3-> [inner sep = 1pt] (X6) at (0, 3) { $\mathit{row}[5]$ };
%             \node <3-> [inner sep = 1pt] (X7) at (0, 2) { $\mathit{row}[6]$ };
%             \node <3-> [inner sep = 1pt] (X8) at (0, 1) { $\mathit{row}[7]$ };
%             \node <3-> [inner sep = 1pt] (X9) at (0, 0) { $\mathit{row}[8]$ };
% 
%             \node <4-> [inner sep = 1pt] (V1) at (6, 8) { $1\vphantom{[]}$ };
%             \node <4-> [inner sep = 1pt] (V2) at (6, 7) { $2\vphantom{[]}$ };
%             \node <4-> [inner sep = 1pt] (V3) at (6, 6) { $3\vphantom{[]}$ };
%             \node <4-> [inner sep = 1pt] (V4) at (6, 5) { $4\vphantom{[]}$ };
%             \node <4-> [inner sep = 1pt] (V5) at (6, 4) { $5\vphantom{[]}$ };
%             \node <4-> [inner sep = 1pt] (V6) at (6, 3) { $6\vphantom{[]}$ };
%             \node <4-> [inner sep = 1pt] (V7) at (6, 2) { $7\vphantom{[]}$ };
%             \node <4-> [inner sep = 1pt] (V8) at (6, 1) { $8\vphantom{[]}$ };
%             \node <4-> [inner sep = 1pt] (V9) at (6, 0) { $9\vphantom{[]}$ };
% 
%             \draw <5> [thick, color=uofgsandstone!50] (X1.east) to [out=0, in=180] (V1.west);
%             \draw <5> [thick, color=uofgsandstone!50] (X1.east) to [out=0, in=180] (V8.west);
%             \draw <5> [thick, color=uofgsandstone!50] (X2.east) to [out=0, in=180] (V2.west);
%             \draw <5> [thick, color=uofgsandstone!50] (X2.east) to [out=0, in=180] (V3.west);
%             \draw <5> [thick, color=uofgsandstone!50] (X3.east) to [out=0, in=180] (V2.west);
%             \draw <5> [thick, color=uofgsandstone!50] (X3.east) to [out=0, in=180] (V3.west);
%             \draw <5> [thick, color=uofgsandstone!50] (X4.east) to [out=0, in=180] (V2.west);
%             \draw <5> [thick, color=uofgsandstone!50] (X4.east) to [out=0, in=180] (V4.west);
%             \draw <5> [thick, color=uofgsandstone!50] (X4.east) to [out=0, in=180] (V5.west);
%             \draw <5> [thick, color=uofgsandstone!50] (X5.east) to [out=0, in=180] (V4.west);
%             \draw <5> [thick, color=uofgsandstone!50] (X5.east) to [out=0, in=180] (V5.west);
%             \draw <5> [thick, color=uofgsandstone!50] (X5.east) to [out=0, in=180] (V6.west);
%             \draw <5> [thick, color=uofgsandstone!50] (X6.east) to [out=0, in=180] (V4.west);
%             \draw <5> [thick, color=uofgsandstone!50] (X6.east) to [out=0, in=180] (V5.west);
%             \draw <5> [thick, color=uofgsandstone!50] (X6.east) to [out=0, in=180] (V6.west);
%             \draw <5> [thick, color=uofgsandstone!50] (X7.east) to [out=0, in=180] (V2.west);
%             \draw <5> [thick, color=uofgsandstone!50] (X7.east) to [out=0, in=180] (V7.west);
%             \draw <5> [thick, color=uofgsandstone!50] (X7.east) to [out=0, in=180] (V9.west);
%             \draw <5> [thick, color=uofgsandstone!50] (X8.east) to [out=0, in=180] (V3.west);
%             \draw <5> [thick, color=uofgsandstone!50] (X8.east) to [out=0, in=180] (V7.west);
%             \draw <5> [thick, color=uofgsandstone!50] (X8.east) to [out=0, in=180] (V8.west);
%             \draw <5> [thick, color=uofgsandstone!50] (X9.east) to [out=0, in=180] (V2.west);
%             \draw <5> [thick, color=uofgsandstone!50] (X9.east) to [out=0, in=180] (V3.west);
%             \draw <5> [thick, color=uofgsandstone!50] (X9.east) to [out=0, in=180] (V5.west);
%             \draw <5> [thick, color=uofgsandstone!50] (X9.east) to [out=0, in=180] (V8.west);
%             \draw <5> [thick, color=uofgsandstone!50] (X9.east) to [out=0, in=180] (V9.west);
% 
%             \draw <6> [thick, color=uofgsandstone!50] (X1.east) to [out=0, in=180] (V8.west);
%             \draw <6> [thick, color=uofgsandstone!50] (X2.east) to [out=0, in=180] (V2.west);
%             \draw <6> [thick, color=uofgsandstone!50] (X3.east) to [out=0, in=180] (V3.west);
%             \draw <6> [thick, color=uofgsandstone!50] (X4.east) to [out=0, in=180] (V2.west);
%             \draw <6> [thick, color=uofgsandstone!50] (X4.east) to [out=0, in=180] (V5.west);
%             \draw <6> [thick, color=uofgsandstone!50] (X5.east) to [out=0, in=180] (V4.west);
%             \draw <6> [thick, color=uofgsandstone!50] (X5.east) to [out=0, in=180] (V5.west);
%             \draw <6> [thick, color=uofgsandstone!50] (X6.east) to [out=0, in=180] (V4.west);
%             \draw <6> [thick, color=uofgsandstone!50] (X6.east) to [out=0, in=180] (V6.west);
%             \draw <6> [thick, color=uofgsandstone!50] (X7.east) to [out=0, in=180] (V2.west);
%             \draw <6> [thick, color=uofgsandstone!50] (X7.east) to [out=0, in=180] (V9.west);
%             \draw <6> [thick, color=uofgsandstone!50] (X8.east) to [out=0, in=180] (V3.west);
%             \draw <6> [thick, color=uofgsandstone!50] (X8.east) to [out=0, in=180] (V7.west);
%             \draw <6> [thick, color=uofgsandstone!50] (X9.east) to [out=0, in=180] (V2.west);
%             \draw <6> [thick, color=uofgsandstone!50] (X9.east) to [out=0, in=180] (V3.west);
%             \draw <6> [thick, color=uofgsandstone!50] (X9.east) to [out=0, in=180] (V5.west);
%             \draw <6> [thick, color=uofgsandstone!50] (X9.east) to [out=0, in=180] (V8.west);
%             \draw <6> [thick] (X9.east) to [out=0, in=180] (V9.west);
%             \draw <6> [thick] (X1.east) to [out=0, in=180] (V1.west);
%             \draw <6> [thick] (X2.east) to [out=0, in=180] (V3.west);
%             \draw <6> [thick] (X3.east) to [out=0, in=180] (V2.west);
%             \draw <6> [thick] (X4.east) to [out=0, in=180] (V4.west);
%             \draw <6> [thick] (X5.east) to [out=0, in=180] (V6.west);
%             \draw <6> [thick] (X6.east) to [out=0, in=180] (V5.west);
%             \draw <6> [thick] (X7.east) to [out=0, in=180] (V7.west);
%             \draw <6> [thick] (X8.east) to [out=0, in=180] (V8.west);
% 
%             \draw <7-8> [thick, ->] ($(X1.east)!0.25!(X1.north east)$) to [out=0, in=180] ($(V8.west)!0.25!(V8.north west)$);
%             \draw <7-8> [thick, ->] ($(X2.east)!0.25!(X2.north east)$) to [out=0, in=180] ($(V2.west)!0.25!(V2.north west)$);
%             \draw <7-8> [thick, ->] ($(X3.east)!0.25!(X3.north east)$) to [out=0, in=180] ($(V3.west)!0.25!(V3.north west)$);
%             \draw <7-8> [thick, ->] ($(X4.east)!0.25!(X4.north east)$) to [out=0, in=180] ($(V2.west)!0.25!(V2.north west)$);
%             \draw <7-8> [thick, ->] ($(X4.east)!0.25!(X4.north east)$) to [out=0, in=180] ($(V5.west)!0.25!(V5.north west)$);
%             \draw <7-8> [thick, ->] ($(X5.east)!0.25!(X5.north east)$) to [out=0, in=180] ($(V4.west)!0.25!(V4.north west)$);
%             \draw <7-8> [thick, ->] ($(X5.east)!0.25!(X5.north east)$) to [out=0, in=180] ($(V5.west)!0.25!(V5.north west)$);
%             \draw <7-8> [thick, ->] ($(X6.east)!0.25!(X6.north east)$) to [out=0, in=180] ($(V4.west)!0.25!(V4.north west)$);
%             \draw <7-8> [thick, ->] ($(X6.east)!0.25!(X6.north east)$) to [out=0, in=180] ($(V6.west)!0.25!(V6.north west)$);
%             \draw <7-8> [thick, ->] ($(X7.east)!0.25!(X7.north east)$) to [out=0, in=180] ($(V2.west)!0.25!(V2.north west)$);
%             \draw <7-8> [thick, ->] ($(X7.east)!0.25!(X7.north east)$) to [out=0, in=180] ($(V9.west)!0.25!(V9.north west)$);
%             \draw <7-8> [thick, ->] ($(X8.east)!0.25!(X8.north east)$) to [out=0, in=180] ($(V3.west)!0.25!(V3.north west)$);
%             \draw <7-8> [thick, ->] ($(X8.east)!0.25!(X8.north east)$) to [out=0, in=180] ($(V7.west)!0.25!(V7.north west)$);
%             \draw <7-8> [thick, ->] ($(X9.east)!0.25!(X9.north east)$) to [out=0, in=180] ($(V2.west)!0.25!(V2.north west)$);
%             \draw <7-8> [thick, ->] ($(X9.east)!0.25!(X9.north east)$) to [out=0, in=180] ($(V3.west)!0.25!(V3.north west)$);
%             \draw <7-8> [thick, ->] ($(X9.east)!0.25!(X9.north east)$) to [out=0, in=180] ($(V5.west)!0.25!(V5.north west)$);
%             \draw <7-8> [thick, ->] ($(X9.east)!0.25!(X9.north east)$) to [out=0, in=180] ($(V8.west)!0.25!(V8.north west)$);
%             \draw <7-8> [thick, <-] ($(X9.east)!0.25!(X9.south east)$) to [out=0, in=180] ($(V9.west)!0.25!(V9.south west)$);
%             \draw <7-8> [thick, <-] ($(X1.east)!0.25!(X1.south east)$) to [out=0, in=180] ($(V1.west)!0.25!(V1.south west)$);
%             \draw <7-8> [thick, <-] ($(X2.east)!0.25!(X2.south east)$) to [out=0, in=180] ($(V3.west)!0.25!(V3.south west)$);
%             \draw <7-8> [thick, <-] ($(X3.east)!0.25!(X3.south east)$) to [out=0, in=180] ($(V2.west)!0.25!(V2.south west)$);
%             \draw <7-8> [thick, <-] ($(X4.east)!0.25!(X4.south east)$) to [out=0, in=180] ($(V4.west)!0.25!(V4.south west)$);
%             \draw <7-8> [thick, <-] ($(X5.east)!0.25!(X5.south east)$) to [out=0, in=180] ($(V6.west)!0.25!(V6.south west)$);
%             \draw <7-8> [thick, <-] ($(X6.east)!0.25!(X6.south east)$) to [out=0, in=180] ($(V5.west)!0.25!(V5.south west)$);
%             \draw <7-8> [thick, <-] ($(X7.east)!0.25!(X7.south east)$) to [out=0, in=180] ($(V7.west)!0.25!(V7.south west)$);
%             \draw <7-8> [thick, <-] ($(X8.east)!0.25!(X8.south east)$) to [out=0, in=180] ($(V8.west)!0.25!(V8.south west)$);
% 
%             \draw <9> [thick, ->, color=uofglawn!80!black] ($(X2.east)!0.25!(X2.north east)$) to [out=0, in=180] ($(V2.west)!0.25!(V2.north west)$);
%             \draw <9> [thick, ->, color=uofglawn!80!black] ($(X3.east)!0.25!(X3.north east)$) to [out=0, in=180] ($(V3.west)!0.25!(V3.north west)$);
%             \draw <9> [thick, <-, color=uofglawn!80!black] ($(X2.east)!0.25!(X2.south east)$) to [out=0, in=180] ($(V3.west)!0.25!(V3.south west)$);
%             \draw <9> [thick, <-, color=uofglawn!80!black] ($(X3.east)!0.25!(X3.south east)$) to [out=0, in=180] ($(V2.west)!0.25!(V2.south west)$);
%             \draw <9> [thick, ->, color=uofgturquoise!80!black] ($(X4.east)!0.25!(X4.north east)$) to [out=0, in=180] ($(V5.west)!0.25!(V5.north west)$);
%             \draw <9> [thick, ->, color=uofgturquoise!80!black] ($(X5.east)!0.25!(X5.north east)$) to [out=0, in=180] ($(V4.west)!0.25!(V4.north west)$);
%             \draw <9> [thick, ->, color=uofgturquoise!80!black] ($(X5.east)!0.25!(X5.north east)$) to [out=0, in=180] ($(V5.west)!0.25!(V5.north west)$);
%             \draw <9> [thick, ->, color=uofgturquoise!80!black] ($(X6.east)!0.25!(X6.north east)$) to [out=0, in=180] ($(V4.west)!0.25!(V4.north west)$);
%             \draw <9> [thick, ->, color=uofgturquoise!80!black] ($(X6.east)!0.25!(X6.north east)$) to [out=0, in=180] ($(V6.west)!0.25!(V6.north west)$);
%             \draw <9> [thick, <-, color=uofgturquoise!80!black] ($(X4.east)!0.25!(X4.south east)$) to [out=0, in=180] ($(V4.west)!0.25!(V4.south west)$);
%             \draw <9> [thick, <-, color=uofgturquoise!80!black] ($(X5.east)!0.25!(X5.south east)$) to [out=0, in=180] ($(V6.west)!0.25!(V6.south west)$);
%             \draw <9> [thick, <-, color=uofgturquoise!80!black] ($(X6.east)!0.25!(X6.south east)$) to [out=0, in=180] ($(V5.west)!0.25!(V5.south west)$);
%             \draw <9> [thick, ->, color=uofgpumpkin!80!black] ($(X7.east)!0.25!(X7.north east)$) to [out=0, in=180] ($(V9.west)!0.25!(V9.north west)$);
%             \draw <9> [thick, ->, color=uofgpumpkin!80!black] ($(X8.east)!0.25!(X8.north east)$) to [out=0, in=180] ($(V7.west)!0.25!(V7.north west)$);
%             \draw <9> [thick, ->, color=uofgpumpkin!80!black] ($(X9.east)!0.25!(X9.north east)$) to [out=0, in=180] ($(V8.west)!0.25!(V8.north west)$);
%             \draw <9> [thick, <-, color=uofgpumpkin!80!black] ($(X9.east)!0.25!(X9.south east)$) to [out=0, in=180] ($(V9.west)!0.25!(V9.south west)$);
%             \draw <9> [thick, <-, color=uofgpumpkin!80!black] ($(X7.east)!0.25!(X7.south east)$) to [out=0, in=180] ($(V7.west)!0.25!(V7.south west)$);
%             \draw <9> [thick, <-, color=uofgpumpkin!80!black] ($(X8.east)!0.25!(X8.south east)$) to [out=0, in=180] ($(V8.west)!0.25!(V8.south west)$);
%             \draw <11-> [thick, <-, color=uofgcobalt] ($(X1.east)!0.25!(X1.south east)$) to [out=0, in=180] ($(V1.west)!0.25!(V1.south west)$);
%             \draw <9-10>  [thick, <-] ($(X1.east)!0.25!(X1.south east)$) to [out=0, in=180] ($(V1.west)!0.25!(V1.south west)$);
%             \draw <9-10>  [thick, ->] ($(X4.east)!0.25!(X4.north east)$) to [out=0, in=180] ($(V2.west)!0.25!(V2.north west)$);
%             \draw <9-10>  [thick, ->] ($(X7.east)!0.25!(X7.north east)$) to [out=0, in=180] ($(V2.west)!0.25!(V2.north west)$);
%             \draw <9-10>  [thick, ->] ($(X8.east)!0.25!(X8.north east)$) to [out=0, in=180] ($(V3.west)!0.25!(V3.north west)$);
%             \draw <9-10>  [thick, ->] ($(X9.east)!0.25!(X9.north east)$) to [out=0, in=180] ($(V2.west)!0.25!(V2.north west)$);
%             \draw <9-10>  [thick, ->] ($(X9.east)!0.25!(X9.north east)$) to [out=0, in=180] ($(V3.west)!0.25!(V3.north west)$);
%             \draw <9-10>  [thick, ->] ($(X9.east)!0.25!(X9.north east)$) to [out=0, in=180] ($(V5.west)!0.25!(V5.north west)$);
%             \draw <9-10>  [thick, ->] ($(X1.east)!0.25!(X1.north east)$) to [out=0, in=180] ($(V8.west)!0.25!(V8.north west)$);
%             \draw <11->  [thick, ->, color=uofgpillarbox] ($(X4.east)!0.25!(X4.north east)$) to [out=0, in=180] ($(V2.west)!0.25!(V2.north west)$);
%             \draw <11->  [thick, ->, color=uofgpillarbox] ($(X7.east)!0.25!(X7.north east)$) to [out=0, in=180] ($(V2.west)!0.25!(V2.north west)$);
%             \draw <11->  [thick, ->, color=uofgpillarbox] ($(X8.east)!0.25!(X8.north east)$) to [out=0, in=180] ($(V3.west)!0.25!(V3.north west)$);
%             \draw <11->  [thick, ->, color=uofgpillarbox] ($(X9.east)!0.25!(X9.north east)$) to [out=0, in=180] ($(V2.west)!0.25!(V2.north west)$);
%             \draw <11->  [thick, ->, color=uofgpillarbox] ($(X9.east)!0.25!(X9.north east)$) to [out=0, in=180] ($(V3.west)!0.25!(V3.north west)$);
%             \draw <11->  [thick, ->, color=uofgpillarbox] ($(X9.east)!0.25!(X9.north east)$) to [out=0, in=180] ($(V5.west)!0.25!(V5.north west)$);
%             \draw <11->  [thick, ->, color=uofgpillarbox] ($(X1.east)!0.25!(X1.north east)$) to [out=0, in=180] ($(V8.west)!0.25!(V8.north west)$);
% 
%             \begin{pgfonlayer}{background}
%                 \node <8-9> [rounded corners, fit = (X1), fill=uofgthistle] {};
%                 \node <8-9> [rounded corners, fit = (V1), fill=uofglavendar] {};
%                 \node <8-9> [rounded corners, fit = (X2) (X3) (V2) (V3), fill=uofglawn] {};
%                 \node <8-9> [rounded corners, fit = (X4) (X5) (X6) (V4) (V5) (V6), fill=uofgcobalt] {};
%                 \node <8-9> [rounded corners, fit = (X7) (X8) (X9) (V7) (V8) (V9), fill=uofgpumpkin] {};
%             \end{pgfonlayer}
%         \end{tikzpicture}
%     }
% 
%     \only<13>{
%         \begin{itemize}
%             \item There's one more special condition that can happen, if we have more values than
%                 domains.
%         \end{itemize}
%     }
% \end{frame}
% 
% \begin{frame}[fragile]{A Sudoku Solver in Choco}
%     \only<1>{\lstinputlisting[basicstyle=\scriptsize\ttfamily]{code/herald20061222E.txt}}
%     \only<2>{\lstinputlisting[language=Java, basicstyle=\scriptsize\ttfamily, keywordstyle=\color{uofgcobalt}]{code/Sudoku-1-snippet.java}}
%     \only<3>{\lstinputlisting[language=Java, basicstyle=\scriptsize\ttfamily, keywordstyle=\color{uofgcobalt}]{code/Sudoku-2-snippet.java}}
%     \only<4>{\lstinputlisting[language=Java, basicstyle=\scriptsize\ttfamily, keywordstyle=\color{uofgcobalt}]{code/Sudoku-3-snippet.java}}
%     \only<5>{\lstinputlisting[language=Java, basicstyle=\scriptsize\ttfamily, keywordstyle=\color{uofgcobalt}]{code/Sudoku-4-snippet.java}}
%     \only<6-7>{
%         \lstinputlisting[language=Java, basicstyle=\scriptsize\ttfamily, keywordstyle=\color{uofgcobalt}]{code/Sudoku-5-snippet.java}
%         \begin{tikzpicture}[remember picture, overlay]
%             \node [anchor=north east, xshift=-0.5cm, yshift=-1cm] at (current page.north east) {
%                 \only<7>{\includegraphics*[keepaspectratio=true,scale=0.45]{eww.jpg}}
%             };
%         \end{tikzpicture}
%     }
%     \only<8>{\lstinputlisting[language=Java, basicstyle=\scriptsize\ttfamily, keywordstyle=\color{uofgcobalt}]{code/Sudoku-6-snippet.java}}
%     \only<9>{\lstinputlisting[language=Java, basicstyle=\scriptsize\ttfamily, keywordstyle=\color{uofgcobalt}]{code/Sudoku-7-snippet.java}}
% \end{frame}
% 
% \begin{frame}[fragile]{Incidentally, 2D arrays in Gecode are Much Nicer}
%     \lstinputlisting[language=C++, basicstyle=\scriptsize\ttfamily, keywordstyle=\color{uofgcobalt}]{code/GecodeSudoku-snippet.cc}
% \end{frame}
% 
% \begin{frame}{Some Experiments}
%     \only<1> {
%         \lstinputlisting[basicstyle=\scriptsize\ttfamily]{code/herald20061222E.txt}
%     }
%     \only<2-3> {
%         \begin{columns}
%             \begin{column}{0.45\textwidth}
%                 \lstinputlisting[basicstyle=\scriptsize\ttfamily]{code/herald20061222E-neq.txt}
%             \end{column}
%             \begin{column}{0.45\textwidth}
%                 \only<3>{
%                     \lstinputlisting[basicstyle=\scriptsize\ttfamily]{code/herald20061222E-alldiff.txt}
%                 }
%             \end{column}
%         \end{columns}
%     }
% 
%     \only<4> {
%         \lstinputlisting[basicstyle=\scriptsize\ttfamily]{code/herald20061222H.txt}
%     }
%     \only<5-6> {
%         \begin{columns}
%             \begin{column}{0.45\textwidth}
%                 \lstinputlisting[basicstyle=\scriptsize\ttfamily]{code/herald20061222H-neq.txt}
%             \end{column}
%             \begin{column}{0.45\textwidth}
%                 \only<6>{
%                     \lstinputlisting[basicstyle=\scriptsize\ttfamily]{code/herald20061222H-alldiff.txt}
%                 }
%             \end{column}
%         \end{columns}
%     }
% 
%     \only<7> {
%         \lstinputlisting[basicstyle=\scriptsize\ttfamily]{code/times20070107.txt}
%     }
%     \only<8-9> {
%         \begin{columns}
%             \begin{column}{0.45\textwidth}
%                 \lstinputlisting[basicstyle=\scriptsize\ttfamily]{code/times20070107-neq.txt}
%             \end{column}
%             \begin{column}{0.45\textwidth}
%                 \only<9>{
%                     \lstinputlisting[basicstyle=\scriptsize\ttfamily]{code/times20070107-alldiff.txt}
%                 }
%             \end{column}
%         \end{columns}
%     }
% 
%     \only<10> {
%         \centering
%         \includegraphics*[keepaspectratio=true,scale=0.24]{hardest.png}
%     }
%     \only<11-12> {
%         \begin{columns}
%             \begin{column}{0.45\textwidth}
%                 \lstinputlisting[basicstyle=\scriptsize\ttfamily]{code/hardest-neq.txt}
%             \end{column}
%             \begin{column}{0.45\textwidth}
%                 \only<12>{
%                     \lstinputlisting[basicstyle=\scriptsize\ttfamily]{code/hardest-alldiff.txt}
%                 }
%             \end{column}
%         \end{columns}
%     }
% 
%     \only<13> {
%         \scalebox{0.7}{
%             \begin{minipage}{2\paperwidth}
%                 \lstinputlisting[basicstyle=\fontsize{5pt}{6pt}\ttfamily]{code/huge.txt}
%         \end{minipage}}
%     }
% 
%     \only<14-15> {
%         \begin{columns}[t]
%             \begin{column}{0.45\textwidth}
%                 \lstinputlisting[basicstyle=\scriptsize\ttfamily]{code/huge-neq.txt}
%             \end{column}
%             \begin{column}{0.45\textwidth}
%                 \only<14> {
%                     \lstinputlisting[basicstyle=\scriptsize\ttfamily\color{white}]{code/huge-alldiff.txt}
%                 }
%                 \only<15> {
%                     \lstinputlisting[basicstyle=\scriptsize\ttfamily]{code/huge-alldiff.txt}
%                 }
%             \end{column}
%         \end{columns}
%     }
% \end{frame}
% 
% \begin{frame}{Do Global Constraints Always Help?}
%     \begin{itemize}
%         \item Sometimes globals make a spectacular difference.
% 
%         \item Sometimes global constraints end up not giving any more deletions than their
%             decompositions, and can take longer to propagate. Sometimes extra deletions don't help
%             anyway.
% 
%         \item Some global constraints cannot be \emph{decomposed}. Any global constraint can be
%             \emph{encoded} using binary constraints in a very unpleasant way involving polynomially
%             many extra variables, but AC on an encoding can be weaker than GAC.
% 
%             \begin{itemize}
%                 \item Difficult homework: $a + b = c$ using binary constraints.
%             \end{itemize}
% 
%         \item Using globals isn't a \emph{guaranteed} benefit, but they make the model easier to
%             read, and it's easier to translate from globals to decompositions and encodings than the
%             other way around.
% 
%             \begin{itemize}
%                 \item Really easy homework: detecting all-different is \NP-hard.
%             \end{itemize}
%     \end{itemize}
% \end{frame}
% 
% \begin{frame}{Some Other Interesting Global Constraints}
% 
%     \begin{itemize}
%         \item All different except 0.
%         \item Global cardinality.
%         \item At least, at most, among.
%         \item $n$ Value.
%         \item Regular.
%     \end{itemize}
% 
% \end{frame}
% 
% \begin{frame}{Are You Smarter than a Constraint Solver?}
%     \only<1> {
%         \begin{center}\begin{tikzpicture}[scale=0.4]
%             \draw[thick, scale=3, color=uofgslate] (0, 0) grid (3, 3);
%             \draw[thick, scale=3, color=uofgslate] (3, 0) grid (9, 1);
% 
%             \node[anchor=center] at (4.5, 1.5) { 34 };
%             \node[anchor=center] at (7.5, 1.5) { 35 };
%             \node[anchor=center] at (7.5, 4.5) { 45 };
%             \node[anchor=center] at (13.5, 1.5) { 345 };
% 
%         \end{tikzpicture}\end{center}
%     }
% 
%     \only<2> {
%         \begin{itemize}
%             \item Remember: propagation only considers \emph{one constraint at a time}, and the only
%                 communication between constraints is by deleting values.
% 
%             \item Automatically combining certain constraints is an active research topic.
%                 \begin{itemize}
%                     \item But getting ``the best possible'' filtering from two ``all different''
%                         constraints simultaneously is \NP-hard\ldots
%                 \end{itemize}
%         \end{itemize}
%     }
% \end{frame}
% 
% \begin{frame}{This is Not The Exam Question}
% 
%     What is a \emph{Hall set}, and why is it useful for propagation? Use the following model to
%     illustrate your answer:
%     \begin{align*}
%         & x_1 \in \{ 4, 5 \}   && x_2 \in \{ 1, 2, 3, 4 \}  && x_3 \in \{ 3, 4, 5 \} \\
%         & x_4 \in \{ 5, 6 \}   && x_5 \in \{ 3, 5 \} \\
%         & \mathrlap{\textit{alldifferent}( x_1, x_2, x_3, x_4, x_5 )}
%     \end{align*}
% 
%     Suppose our solver did not have an ``all different'' constraint. Show how to rewrite this model
%     using only binary constraints.  What effect would this have on propagation? \\[0.2cm]
% 
%     Aside from propagation, describe another benefit of global constraints.
% 
% \end{frame}
% 
% \begin{frame}[plain,noframenumbering]
%     \begin{tikzpicture}[remember picture, overlay]
%         \node at (current page.north west) {
%             \begin{tikzpicture}[remember picture, overlay]
%                 \fill [fill=uofguniversityblue, anchor=north west] (0, 0) rectangle (\paperwidth, -1.7cm);
%             \end{tikzpicture}
%         };
% 
%         \node (logo) [anchor=north east, shift={(-0.3cm,-0.2cm)}] at (current page.north east) {
%             \includegraphics*[keepaspectratio=true,scale=0.55]{UoG_keyline.pdf}
%         };
%     \end{tikzpicture}
% \end{frame}
% 
\end{document}


